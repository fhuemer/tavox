\documentclass[aspectratio=169]{beamer}
\usepackage{tavox}


\usepackage{ifthen}

\ifdefined\showtranscript
\else
	\def\showtranscript{0}
\fi

\ifthenelse{\equal{\showtranscript}{1}}{
	\setbeameroption{show notes on second screen=top}
}{}


\begin{document}

\begin{frame}[fragile]

	\speak<1>[predelay=1.0]{
		With tavox you can also use overlay specifications.
	}

	\speak<2>[predelay=0.5]{
		To do so simply use the usual angle bracket syntax.
	}

	\speak<3>[predelay=0.5]{
		Here you can see an example how a speak command looks like, that shall only be active at the third overlay step and speak the word "hello".
	}

	\speak<4>[predelay=0.5]{
		It is also possible to use a range in the overlay specification. In such a case the text is spoken over all steps in the range. Every step is shown for the same amount of time.
	}

	\speak<5-7>[predelay=0.5]{
		This example uses a range from 5 to 7.
	}

	\speak<8>[predelay=0.5]{
		Keep in mind that if you use overlay specifications for a slide then every "speak" command on that slide should use one.
		A speak command without one causes the specified text to be spoken at every overlay step, which is probably not something often required.
	}

	\speak<8>[predelay=0.5]{
		Please, have a look into the source files of this presentation to learn more.
	}

	\begin{itemize}
		\item<1-> \texttt{tavox} supports overlay specification
		\item<2-> usual overlay syntax (\texttt{< >})
		\item<3-> example: say ``Hello'' on the third overlay step: \verb|\speak<3>{Hello}|
		\item<4-> ranges are also possible (example: \verb|\speak<4-5>{Hello}|)
		\begin{itemize}
			\item<5-> Step 5
			\item<6-> Step 6
			\item<7-> Step 7
		\end{itemize}
		\item<8-> if overlay specifications are used on a slide, all speak commands should have one
	\end{itemize}
\end{frame}

\end{document}

